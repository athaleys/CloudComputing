\section{Apatar}
Apatar~\cite{hid-sp18-402-www-apatar} is a data integration tool which
provides the capability to work with data across different systems and
helps to move data between those systems. It also provides ETL
capability for the data extraction and transformation. Application
point of view it can be used in data warehousing, data migration,
synchronization and integration between applications. It can be used
across heterogeneous systems like databases, files, FTP, Queue, and
applications like ERP, CRM. Since it is an open source tool developed
in Java, it provides platform independence and can be used on any
operating system. It provides flexible deployment options as desktop,
server or embedded into a 3rd party software. The desktop deployment
comes with a GUI client installation along with command line support
on the local machine. Server deployment allows Apatar to be deployed
as server engine over the network. The embedded option allows other
software providers to embed Apatar into their software to provide data
integration capabilities. Apatar has GUI for mapping and design which
can be used by technical as well as the non-technical person. Apatar
is based on modular open application architecture which allows
customization and flexibility to modify the source code for customized
business logic or integration with new systems. As per the Apatar
website, it currently supports connectivity and works with Oracle, MS
SQL, MySQL, Sybase, DB2, MS Access, PostgreSQL, XML, InstantDB,
Paradox, BorlandJDataStore, CSV, MS Excel, Qed, HSQL, Compiere ERP,
SalesForce.Com, SugarCRM, Goldmine, any JDBC data sources and
more. Apatar also has data quality tool which helps with the data
cleansing. It provides support to multiple languages as it is Unicode
compliant. The Apatar architecture consists of 3 major component as
presentation/GUI, ETL, and data source. GUI is used to perform various
data integration task like data mapping, data source configuration etc
in a user-friendly way. Data source provides various connectors to
connect with different data sources like databases, files,
application (SAP, Siebel, etc), real-time feeds like queuing
services. Extract, Transformation and Load (ETL) component provides
functionality like data transformation, real-time in-memory data
processing, data cleansing and validation, data exception/rejection
management, data loading, post data load processing like archival,
indexing, aggregation and scheduling and event management.
